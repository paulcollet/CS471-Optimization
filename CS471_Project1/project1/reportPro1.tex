%----------------------------------------------------------------------------------------
%	PACKAGES AND OTHER DOCUMENT CONFIGURATIONS
%----------------------------------------------------------------------------------------

\documentclass[paper=a4, fontsize=11pt]{scrartcl} % A4 paper and 11pt font size

\usepackage{csvsimple}
\begin{filecontents*}{benchmark.csv}

\end{filecontents*}

\usepackage[T1]{fontenc} % Use 8-bit encoding that has 256 glyphs
\usepackage{fourier} % Use the Adobe Utopia font for the document - comment this line to return to the LaTeX default
\usepackage[english]{babel} % English language/hyphenation
\usepackage{amsmath,amsfonts,amsthm} % Math packages

\usepackage{lipsum} % Used for inserting dummy 'Lorem ipsum' text into the template

\usepackage{sectsty} % Allows customizing section commands
\usepackage{rotating} % for long tables
\allsectionsfont{\centering \normalfont\scshape} % Make all sections centered, the default font and small caps

\usepackage{fancyhdr} % Custom headers and footers
\pagestyle{fancyplain} % Makes all pages in the document conform to the custom headers and footers
\fancyhead{} % No page header - if you want one, create it in the same way as the footers below
\fancyfoot[L]{} % Empty left footer
\fancyfoot[C]{} % Empty center footer
\fancyfoot[R]{\thepage} % Page numbering for right footer
\renewcommand{\headrulewidth}{0pt} % Remove header underlines
\renewcommand{\footrulewidth}{0pt} % Remove footer underlines
\setlength{\headheight}{13.6pt} % Customize the height of the header

\numberwithin{equation}{section} % Number equations within sections (i.e. 1.1, 1.2, 2.1, 2.2 instead of 1, 2, 3, 4)
\numberwithin{figure}{section} % Number figures within sections (i.e. 1.1, 1.2, 2.1, 2.2 instead of 1, 2, 3, 4)
\numberwithin{table}{section} % Number tables within sections (i.e. 1.1, 1.2, 2.1, 2.2 instead of 1, 2, 3, 4)

\setlength\parindent{0pt} % Removes all indentation from paragraphs - comment this line for an assignment with lots of text
%----------------------------------------------------------------------------------------
%	TITLE
%----------------------------------------------------------------------------------------

\newcommand{\horrule}[1]{\rule{\linewidth}{#1}} % Create horizontal rule command with 1 argument of height

\title{	
\normalfont \normalsize 
% Your university, school and/or department name(s)
\textsc{Central Washington University, Department of Computer Science} \\ [25pt] 
\horrule{0.5pt} \\[0.4cm] % Thin top horizontal rule
\huge CS471 Project1 \\ % The assignment title
\horrule{2pt} \\[0.5cm] % Thick bottom horizontal rule
}

\author{Paul O. Collet} % Your name

\date{\normalsize\today} % Today's date or a custom date

\begin{document}

\maketitle % Print the title

%----------------------------------------------------------------------------------------
%	ABOUT PROGRAM
%----------------------------------------------------------------------------------------
\section{About the Program}

This program was created a project requirement for CS471: Optimization. It runs fifteen standard benchmark functions and prints the results onto a file called, "benchmark.csv". The results contain     the function name, the dimension used, the results of each try, the mean average, the range, the median, and the time in seconds. See Table 2.1 for analysis. NOTE: The numbers of tries is determined by the variable MINSTAT which might need to be increased on faster machines.



%----------------------------------------------------------------------------------------
%	FUNCTIONS
%----------------------------------------------------------------------------------------
\section{Functions Used}

\begin{enumerate}
\item Schwefel's function
\item 1st De Jong's function
\item Rosenbrock
\item Rastrigin
\item Griewangk
\item Sine Envelope Sine Wave (SESW)
\item Stretched V Sine Wave (SVSW)
\item Ackley's One
\item Ackley's Two
\item Egg Holder
\item Rana
\item Pathological
\item Michalewicz
\item Masters Cosine Wave
\item Shekal's Foxholes


\end{enumerate}
%----------------------------------------------------------------------------------------
%	TABLES(S)
%----------------------------------------------------------------------------------------

\begin{sidewaystable}
	\small
	\centering
	\caption{Comparison of 10, 20 and 30 dimensions}
	\label{Tab1}
	\begin{tabular}{c|ccccc|ccccc|ccccc}    
	\noalign{\smallskip}\hline\noalign{\smallskip}
	Dimensions 	& \multicolumn{5}{c}{10}
				& \multicolumn{5}{|c|}{20}
				& \multicolumn{5}{c}{30} \\ 
	\noalign{\smallskip}\hline\noalign{\smallskip}
		 & Avg & Median & Range & SD & T(s) 
		 & Avg & Median & Range & SD & T(s) 
		 & Avg & Median & Range & SD & T(s) \\ 
\noalign{\smallskip}\hline\noalign{\smallskip}
			$f_1$ & 2.40 & 0.57 & 5.36 & 1.48 & 0.6 & 2.34 & 2.16 & 2.39 & 0.08 & 1 & 2.38 & 2.36 & 2.4 & 0.02 & 1 \\ 
			$f_2$ & 3.30 & 2.13 & 4.48 & 0.72 & 1.08 & 3.25 & 3.15 & 3.3 & 0.06 & 2 & 3.29 & 3.24 & 3.3 & 0.02 & 2 \\ 
			$f_3$ & 4.81 & 0.97 & 7.45 & 1.77 & 2.31 & 4.15 & 3.73 & 4.61 & 0.28 & 1.25 & 4.24 & 3.88 & 4.67 & 0.25 & 1.25 \\ 
			$f_4$ & 6.23 & 4.86 & 7.36 & 0.89 & 3.54 & 5.36 & 4.94 & 5.83 & 0.28 & 2.5 & 5.75 & 5.43 & 6.12 & 0.23 & 2.5 \\ 
			$f_5$ & 6.62 & 4.48 & 8.40 & 1.42 & 4.63 & 5.55 & 5.25 & 5.87 & 0.2 & 5 & 6.03 & 5.74 & 6.34 & 0.2 & 5 \\ 
			$f_6$ & 2.24 & 1.00 & 3.74 & 0.95 & 8.31 & 0.37 & 0.03 & 0.79 & 0.24 & 2.5 & 1.42 & 1.04 & 1.86 & 0.26 & 2.5 \\ 
			$f_7$ & 5.90 & 4.67 & 7.94 & 0.93 & 17.08 & 3.9 & 3.59 & 4.25 & 0.21 & 5 & 5.17 & 4.92 & 5.56 & 0.21 & 5 \\ 
			$f_8$ & 5.14 & 4.00 & 6.01 & 0.62 & 28.42 & 3.62 & 3.36 & 3.88 & 0.16 & 10 & 4.68 & 4.39 & 5.01 & 0.19 & 10 \\ 
			$f_9$ & 4.03 & 2.56 & 5.49 & 1.09 & 195.33 & 1.29 & 1.04 & 1.58 & 0.17 & 10 & 3.09 & 2.8 & 3.47 & 0.2 & 10 \\ 
			$f_{10}$ & 3.99 & 2.82 & 4.81 & 0.66 & 243.33 & 2.17 & 1.99 & 2.35 & 0.11 & 20 & 3.57 & 3.31 & 3.86 & 0.17 & 20 \\ 
			$f_{11}$ & 3.23 & 2.53 & 4.03 & 0.46 & 435.34 & 1.19 & 1.08 & 1.34 & 0.08 & 50 & 2.47 & 2.16 & 2.78 & 0.2 & 50 \\ 
			$f_{12}$ & 3.23 & 2.53 & 4.03 & 0.46 & 435.34 & 1.19 & 1.08 & 1.34 & 0.08 & 50 & 2.47 & 2.16 & 2.78 & 0.2 & 50 \\ 
			$f_{13}$ & 3.23 & 2.53 & 4.03 & 0.46 & 435.34 & 1.19 & 1.08 & 1.34 & 0.08 & 50 & 2.47 & 2.16 & 2.78 & 0.2 & 50 \\ 
			$f_{14}$ & 3.23 & 2.53 & 4.03 & 0.46 & 435.34 & 1.19 & 1.08 & 1.34 & 0.08 & 50 & 2.47 & 2.16 & 2.78 & 0.2 & 50 \\ 
			$f_{15}$ & 3.23 & 2.53 & 4.03 & 0.46 & 435.34 & 1.19 & 1.08 & 1.34 & 0.08 & 50 & 2.47 & 2.16 & 2.78 & 0.2 & 50 \\ 
\noalign{\smallskip}\hline\noalign{\smallskip}

	\end{tabular}
\end{sidewaystable}

%----------------------------------------------------------------------------------------
%	ANALYSIS/CONCLUSION
%----------------------------------------------------------------------------------------
\section{Conclusion}

Overall, the benchmarks work very quickly. Some observations:
\begin{itemize}
\item The Schwefels function has the greatest range but with low averages.
\item The Rosenbrock had the largest sums but the range remained between 150 and 200.
\item The RosenBrock, SESW, Ackley's One, and MCW were measured as the slowest. 
\item Shekal's Foxholes had the smallest averages.
\end{itemize}


%----------------------------------------------------------------------------------------
\end{document}